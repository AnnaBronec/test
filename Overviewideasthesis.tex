\documentclass[12pt, a4paper, openany]{article}
\usepackage[left=3cm, top=3cm, bottom=3cm, right=4cm]{geometry}
\DeclareUnicodeCharacter{00A0}{}

\usepackage{secdot} % Dots in Section Numbers
\usepackage[utf8]{inputenc}
\usepackage[T1]{fontenc}
\usepackage{amsmath}
\usepackage{amssymb}
\usepackage{graphicx}
\usepackage{hyperref}

\title{
  {\textbf{Thoughts and ideas for a thesis}}\\
}

\begin{document}

\maketitle

\section{Do psychedelic drugs infect l6b "more" than other layers?}
\begin{itemize}
\setlength\itemsep{0em}
\item therefor compare different cells from different layers under the influence of psychedelics 
\item is there anything extraordinary within l6b while inducing psychedelics
\item it would be interesting to see the influence on the other layers without l6b (not sure if that is possible)
\item compare different types of psychedelics (Ketamine, DMT, Salvinorin..)
\item I guess this could be the starting point for the other questions 
\end{itemize}
\textbf{Methods}:  Mapping with CaMPARI, Patch-clamp, document with confocal microscope

\section{Is orexin involved in the effect of psychedelics on the neo cortex?}
\begin{itemize}
\setlength\itemsep{0em}
\item as we know orexin only effects l6b cells
\item therefor it would be interesting to see how the effect of psychedelics would behave within and without orexin 
\item orexin comes from the hypothalamus
\item to measure the effect we could use an Antagonist or block the hypothalamus 
\item it is known, that psychedelics change the appetite of people as well as the sleeping rhythm, is that due to the effect of change in level of orexin?
\item compare DRD1a cells with CTGF cells in l6b under the influence of Orexin 
\item question: What I was wondering is that l6b gets only a small input of the thalamus but is the only receptor for orexin and orexin comes from the thalamus. Does input only mean the electrical (sensory) impulse and not the transport of neurotransmitters?
\item Orexin also seems to have an impact in stressful conditions, therefor the comparison or context with DMT would be very interesting 
\item Orexin also seems to play a role in anxiety studies 
\end{itemize} 
\textbf{Methods}: Mapping with CaMPARI (active brainstates), Patch-clamp, block the orexin receptors with dual orexin receptor antagonists DORA (N-biphenyl-2-yl-1-[(1-methyl-1H-benzimidazol-2yl)sulfanyl] acetyl-L-prolinamide), use DRD1a mouseline and CTGF mouseline and compare with and without orexin, document with confocal microscope\\


\section{Psychedelics link up the brain. Is this related to the fact that l6b has a lot of long range neurons?}
\begin{itemize}
\setlength\itemsep{0em}
\item check the effect of psychedelics on long range neurons 
\item are they extra ordinary active or different during the influence of psychedelics?
\item try to block the connection with long-range neurons and measure the effect 
\item this would only make sense if we would know that long-range neurons amplify the connectivity in the brain 
\item are there any other parameters with whom we could check the connectivity of the brain during the influence of psychedelics? 

\end{itemize} 
\textbf{Methods}:  Mapping with CaMPARI, Patch-clamp, document with confocal microscope, is there a mousline for long-range neurons?, 

\section{Compare L4 and L6b cells under the influence of psychedelics}
\begin{itemize}
\setlength\itemsep{0em}
\item this could be interesting because they act contrary to each other within the thalamocortical neurons and the long-range input 
\item could help to characterize the functions of the layers more detailed even though as the functions of the cells and psychedelics
\item how important is the effect of long-range neurons and thalamocortical interactions for the effect of psychedelics 
\item if the data analyses is not too difficult this would be interesting to mix up with the question from 3.
\item question: Was there a an impact on L4 which was extraordinary low? That could suggest that the thalamocortical and longe range influence have an impact on the acting of the psychedelics
\end {itemize}
\textbf{Methods}:  Mapping with CaMPARI, Patch-clamp, document with confocal microscope


\section{Is the inner connectivity of l6b a reason for the impact of psychedelics?}
\begin{itemize}
\setlength\itemsep{0em}
\item due to the fact, that l6b cells have a strong di-synpatic inhibition it would be interesting to see how this acts under the influence of psychedelics
\item we could try to decouple the inner connectivity and see the effect
\end {itemize}

\section{Claustrum and psychedelics}
\begin{itemize}
\setlength\itemsep{0em}
\item could also be interesting so see how the claustrum behaves during psychedelics 
\end{itemize}
\textbf{Methods}: Mapping with CaMPARI, Patch-clamp, document with confocal microscope
\end{document}